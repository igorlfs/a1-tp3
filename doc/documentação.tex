\documentclass{article}

\usepackage{enumitem}
\usepackage{indentfirst}
\usepackage{amsfonts}
\usepackage{hyperref}           % Use links
\usepackage{graphicx}           % Coloque figuras
\usepackage{float}              % [...] no lugar adequado!
\usepackage[T1]{fontenc}        % Encoding para português 
\usepackage{lmodern}            % Conserta a fonte para PT
\usepackage[portuguese]{babel}  % Português
\usepackage{hyphenat}           % Use hifens corretamente
\usepackage[table]{xcolor}
\usepackage{algorithm}
\usepackage[noend]{algpseudocode}

\graphicspath{{./img/}}
\hyphenation{mate-mática recu-perar}
\setlist{  
    listparindent=\parindent,
    parsep=0pt,
}

\def\code#1{\texttt{#1}}

\makeatletter
\def\BState{\State\hskip-\ALG@thistlm}
\makeatother


\author{\textbf{Igor Lacerda Faria da Silva}}
\title{{TP 03 - Trabalho Prático 3}

Algoritmos I}
\date{%
    \href{mailto:igorlfs@ufmg.br}{\nolinkurl{igorlfs@ufmg.br}}
}
\begin{document}
\maketitle

\section{Introdução}

O TP foi desenvolvido em C++, em ambiente Linux. Em termos de boas práticas foram as usadas as ferramentas \code{clang-format} e \code{clang-tidy} para garantir um código limpo e consistente.

\section{Modelagem Computational do Problema}

O programa lê um mapa de uma planta de uma residência e um conjunto de tamanhos (comprimento, largura) de mesas, e busca encaixar a maior mesa (em termos de área, com desempate de largura) possível dadas as restrições de espaço da casa (alguns locais podem estar ocupados). Após a leitura da entrada, as mesas são ordenadas pelos critérios especificados e é realizada uma busca sequencial, verificando se na casa existe alguma região capaz de comportar cada mesa. Quando é encontrada uma mesa que cabe, o programa retorna suas dimensões.

A busca percorre toda a planta (excluindo as bordas e possíveis locais que ``estouram'' a casa), considerando cada elemento da matriz como um candidato (correspondendo à posição de canto superior esquerdo). Se no espaço que começa nessa posição existe um ``bloco'' ocupado, o candidato é descartado e é checado o bloco seguinte, e assim por diante. Em sucessão é testada a mesma mesa, mas com as dimensões invertidas. É ``necessário''\footnote{É claro que existe de se resolver o problema que não faz uso dessa \textit{gambiarra}.} fazer essa busca exaustiva porque dependendo do arranjo da casa, existem diversas maneiras de se expandir um retângulo que conteria a mesa. Em particular, isso se deve ao fato de que as \textit{expansões máximas} dos retângulos podem se sobrepor.

Embora essa abordagem não seja eficiente, ela está correta, e não é difícil ver isso. Suponha que o algoritmo descrito acima dê a resposta errada para alguma entrada, ou seja, que existe uma mesa \textit{maior} que a encontrada pelo algoritmo. Se ela é de fato maior, então foi checada antes da encontrada, pois as mesas são ordenadas. E como o algoritmo não parou nela, isso significa que ela não cabe (toda a matriz foi checada), o que é absurdo.

\section{Estruturas de Dados e Algoritmos}

Como mencionado anteriormente, a planta é representada por uma matriz (um vetor de vetores de \code{char}'s), que na implementação é a classe \code{Map}, que é responsável por fazer a leitura da planta e por buscar. O outro algoritmo importante é a ordenação, com uma função de comparação que atende à especificação.

Uma descrição em alto nível do programa seria: ordene as mesas, para cada mesa, veja se ela cabe no retângulo que tem como elemento superior esquerdo (0,0). Se não for, procure na ``expansão'' (0,1) e continue assim, até esgotar o comprimento da casa. Prossiga para a linha seguinte, até esgotar todas as linhas. Se ainda não der certo, inverta as dimensões da mesa. Vá para a mesa seguinte se a mesa atual não couber\footnote{Creio eu que, com uma ideia tão simples, não é necessário fazer um pseudocódigo.}.

\section{Análise de Complexidade de Tempo}

A leitura da entrada tem complexidade \( \Theta(mn + k) \) em que \( m, n \) são as dimensões da casa e \( k \) é o número de mesas candidatas. Depois, a ordenação das mesas tem complexidade \( k \log k \). A busca de uma mesa para uma posição especifica tem complexidade \( \Theta(ij) \) em que \( i, j \) são as dimensões da mesa. Tomando \( i_0 = \max i \) e \( j_0 = \max j \) como as dimensões máximas, podemos avaliar a complexidade da busca como um todo: para uma cada das \( k \) mesas, são checadas \( mn \) posições em que se confere no máximo \( i_0j_0 \) caracteres, no pior caso. Concluímos então que a complexidade da busca é \( \Theta(kmni_0j_0) \) e do programa como um todo é \( \Theta(k \log k + kmni_0j_0) \).

\section{Compilação}

\texttt{g++ src/map.cpp -c -Wall -std=c++17 -Ilib -o obj/map.o}

\texttt{g++ src/main.cpp -c -Wall -std=c++17 -Ilib -o obj/main.o}

\texttt{g++ obj/map.o obj/main.o -o tp3}

\end{document}
